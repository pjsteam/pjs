
%%%%%%%%%%%%%%%%%%%%%%% file typeinst.tex %%%%%%%%%%%%%%%%%%%%%%%%%
%
% This is the LaTeX source for the instructions to authors using
% the LaTeX document class 'llncs.cls' for contributions to
% the Lecture Notes in Computer Sciences series.
% http://www.springer.com/lncs       Springer Heidelberg 2006/05/04
%
% It may be used as a template for your own input - copy it
% to a new file with a new name and use it as the basis
% for your article.
%
% NB: the document class 'llncs' has its own and detailed documentation, see
% ftp://ftp.springer.de/data/pubftp/pub/tex/latex/llncs/latex2e/llncsdoc.pdf
%
%%%%%%%%%%%%%%%%%%%%%%%%%%%%%%%%%%%%%%%%%%%%%%%%%%%%%%%%%%%%%%%%%%%


\documentclass[runningheads,a4paper]{llncs}

\usepackage{amssymb}
\setcounter{tocdepth}{3}
\usepackage{graphicx}

\usepackage{url}
\urldef{\mailsa}\path|{dschenkelman, mservetto}@fi.uba.ar|
\newcommand{\keywords}[1]{\par\addvspace\baselineskip
\noindent\keywordname\enspace\ignorespaces#1}

\begin{document}

\mainmatter  % start of an individual contribution

% first the title is needed
\title{Using algorithmic skeletons in EcmaScript to parallelize computation in browsers through Web Workers}

% a short form should be given in case it is too long for the running head
\titlerunning{Algorithmic skeletons to parallelizing computation in browsers}

% the name(s) of the author(s) follow(s) next
%
% NB: Chinese authors should write their first names(s) in front of
% their surnames. This ensures that the names appear correctly in
% the running heads and the author index.
%
\author{Damian Schenkelman\and Matias Servetto}
%
\authorrunning{Algorithmic skeletons to parallelizing computation in browsers}
% (feature abused for this document to repeat the title also on left hand pages)

% the affiliations are given next; don't give your e-mail address
% unless you accept that it will be published
\institute{Buenos Aires University, School of Engineering,\\
Paseo Colon 850, Buenos Aires, Argentina\\
\mailsa\\
\url{http://www.fi.uba.ar/}}

%
% NB: a more complex sample for affiliations and the mapping to the
% corresponding authors can be found in the file "llncs.dem"
% (search for the string "\mainmatter" where a contribution starts).
% "llncs.dem" accompanies the document class "llncs.cls".
%

\toctitle{Lecture Notes in Computer Science}
\tocauthor{Authors' Instructions}
\maketitle


\begin{abstract}
With the growing tendency of browsers being used for CPU intensive applications (such as 3D games) it is important to take advantage of multiple cores to reduce user perceived latency. This paper presents a JavaScript library based on algorithmic skeletons that allows users to take advantage of parallel processing in browsers while maintaining a familiar coding style.
\keywords{browsers, ecmascript, parallelization, algorithmic skeletons}
\end{abstract}

\section{Introduction}

In 1965 Gordon Moore proposed the popularly known ``Moore's law" \cite{moore65}, which predicts that the number of transitors in integrated circuits will double every two years.
As a consequence, programs that use a single processor/core will automatically become faster without the need to modify them with that purpose in mind.

In the year 2010 it was predicted that this tendency would be slowly reaching its end, mostly due to reasons related to heat dissipation. This is the cause why newer computer have a greater number of processors instead of processors with more computing power.

To be able to maximize the performance of these computers, it is paramount that systems professionals can create programs that process data in parallel, through different mechanisms, such as threads or processes executing simultaneously in different processors. In the case of application executed on a single device (e.g. browsers and web pages, applications for mobile devices, desktop apps, server applications) a lot of languages and platforms provide a simple way for programmers to abstract the complexity and coordination required for this type of processing. For example, the .NET platform has the Task Parallel Library (TPL) and Parallel LINQ (PLINQ) \cite{ms-par} libraries that use algorithmic skeletons modelled with high order functions so developers can perform complex operations through a simple API.

On a different note, one of the tendencies that is beggining to take off is the use of EcmaScript \cite{es-web} (also known as Javascript) to create applications that were previously only considered viable in a native environment, such as video games. This has been possible thanks to components like asm.js \cite{asm-web} and WebGL \cite{webgl-web} together with the constant evolution of browsers and JavaScript runtime engines.
In this context, one of the plans of the commitee that develops the language is to provide an API to simplify the processing of data in parallel for version 7 (ES7) of the language, with the goal of keeping up with native applications. Initiatives such as ParallelJS \cite{par-js} and River trail \cite{rivertrail} or the possibility of taking advantage of SIMD instructions \cite{js-simd} (single instruction multiple data) are some of the options to implement this.

The goal of this paper is to provide an alternative for parallel code execution in EcmaScript, through a library that exposes high order functions to model algorithmic skeletons such as map, filter and reduce. This library will take advantage of Web Workers \cite{w3c-ww} (the mechanism proposed by browsers for parallelism at the time of writing) to allow the simultaneous execution of code in multiple processors.

\section{Serialization and Transference}
Let's consider an array of \verb+N+ elements on which a particular transformation is to be performed through the \verb+map+ function. If the transformation is executed using a single thread (no parallelism) and the average time to process each element is \verb+t+ then the total time (\(T_{ser}\)) for the operation can be approximated as:

\begin{equation}
  T_{ser} = \sum_{i = 0}^{N}t = Nt \;  .
\end{equation}

When trying to parallelize this operation using \(K\) threads the ideal goal is to reach a total time (\(T_{par}\)) that is:

\begin{equation}
  T_{par} = \frac{T_{ser}}{K} \;  .
\end{equation}

Nevertheless, there are additional time consuming tasks other than the main computation that need to be considered when performing the operation in parallel in programs where not all memory is shared. These are:
\begin{itemize}
  \item Serializing/deserializing the elements to transfer.
  \item Transferring the elements back and forth between the UI thread and the workers.
  \item Transferring to each worker the functions for the transformation.
\end{itemize}

If we consider \(T_{ft}\) as the function transfer time, \(T_{et}\) as the elements transfer time, and \(T_{s}\) as the serialization/deserialization time it is clear that:

\begin{equation}
  T_{sync} = T_{ft} + T_{et} + T_{s} \;  .
\end{equation}

\begin{equation}
  T_{par} \approx \frac{T_{ser}}{K} + T_{sync} \;  .
\end{equation}

Based on the previous equations it can be deduced that the more \(T_{sync}\) can be reduced, the closer to the ideal scenario the computation will be.

In our case we are trying to transfer objects between a browser's JavaScript UI thread and Web Workers so we are constrained by the means of that environment. The Worker interface is the following \cite{w3c-ww}:
\medskip

\noindent
{\it Web Worker interface}
\begin{verbatim}
[Constructor(DOMString scriptURL)]
interface Worker : EventTarget {
  void terminate();

  void postMessage(any message, optional sequence<Transferable> transfer);
  [TreatNonCallableAsNull] attribute Function? onmessage;
};
Worker implements AbstractWorker;
\end{verbatim}
%
\noindent
{\small (Example extracted from the W3C Web Workers specification)}

As the aforementioned interface states one can either send just a message or send a message with a sequence of transferable objects. From section \textbf{2.7.5} of the HTML Standard \cite{html-whatwg}:
\textit{``Some objects support being copied and closed in one operation. This is called transferring the object, and is used in particular to transfer ownership of unsharable or expensive resources across worker boundaries."}

\subsection{Elements}
Considering the definition of a \verb+Transferable+ the latter seems like a good alternative to minimize both \(T_{et}\) and \(T_{s}\). Even more so when one considers that otherwise objects are copied using structured cloning(explained in section \textbf{2.7.6} of that same standard).

To verify our hypothesis, we put together two benchmarks\footnote{All benchmarks in this document use the environment described in Tab.~\ref{tab:env}.} that aim to verify the difference between invoking \verb+postMessage+ with and without structured cloning for a \verb+SharedTypedArray+. Both transfer a \verb+SharedTypedArray+ back and forth between the UI thread and a worker; the only difference between the two is that one has 100000 (a hundred thousand) elements in the \verb+SharedTypedArray+ and the other one 1000000 (a million). The results are in Tab.~\ref{tab:transf-vs-clone}. As it can be seen from the  results, increasing the amount of elements by 10x does not change the amount of operations that can be performed when using \verb+Transferable+ objects, it scales. On the other hand, the amount of operations that can be performed with structured cloning greatly decreases.

For that reason our library only works with \verb+SharedTypedArray+s and \verb+TypedArrays+.

\begin{table}
  \centering
  \begin{tabular}{|l|l|}
    \hline
    Computer & Mac Book Air \\
    Processor & Intel Core i5 \\
    Clock Frequency & 1.8 GHz \\
    System Memory & 4 GB \\
    Operating System & OS X Yosemite 10.10.3 \\
    Browser version & Firefox Nightly 40.0a1 (2015-04-26) \\
    \hline
  \end{tabular}
  \caption{Benchmarks environment.}
  \label{tab:env}
\end{table}

\begin{table}
  \centering
  \begin{tabular}{|c|c|c|}
    \hline
    Elements & Cloning [ops/sec] & Transferring [ops/sec] \\
    \hline
    \(1E5\) & 1000 & 9000 \\
    \(1E6\) & 100 & 16000 \\
    \hline
  \end{tabular}
  \caption{Difference between transferring and cloning shared buffers.}
  \label{tab:transf-vs-clone}
\end{table}

\section{Paper Preparation}

Springer provides you with a complete integrated \LaTeX{} document class
(\texttt{llncs.cls}) for multi-author books such as those in the LNCS
series. Papers not complying with the LNCS style will be reformatted.
This can lead to an increase in the overall number of pages. We would
therefore urge you not to squash your paper.

Please always cancel any superfluous definitions that are
not actually used in your text. If you do not, these may conflict with
the definitions of the macro package, causing changes in the structure
of the text and leading to numerous mistakes in the proofs.

If you wonder what \LaTeX{} is and where it can be obtained, see the
``\textit{LaTeX project site}'' (\url{http://www.latex-project.org})
and especially the webpage ``\textit{How to get it}''
(\url{http://www.latex-project.org/ftp.html}) respectively.

When you use \LaTeX\ together with our document class file,
\texttt{llncs.cls},
your text is typeset automatically in Computer Modern Roman (CM) fonts.
Please do
\emph{not} change the preset fonts. If you have to use fonts other
than the preset fonts, kindly submit these with your files.

Please use the commands \verb+\label+ and \verb+\ref+ for
cross-references and the commands \verb+\bibitem+ and \verb+\cite+ for
references to the bibliography, to enable us to create hyperlinks at
these places.

For preparing your figures electronically and integrating them into
your source file we recommend using the standard \LaTeX{} \verb+graphics+ or
\verb+graphicx+ package. These provide the \verb+\includegraphics+ command.
In general, please refrain from using the \verb+\special+ command.

Remember to submit any further style files and
fonts you have used together with your source files.

\subsubsection{Headings.}

Headings should be capitalized
(i.e., nouns, verbs, and all other words
except articles, prepositions, and conjunctions should be set with an
initial capital) and should,
with the exception of the title, be aligned to the left.
Words joined by a hyphen are subject to a special rule. If the first
word can stand alone, the second word should be capitalized.

Here are some examples of headings: ``Criteria to Disprove
Context-Freeness of Collage Language", ``On Correcting the Intrusion of
Tracing Non-deterministic Programs by Software", ``A User-Friendly and
Extendable Data Distribution System", ``Multi-flip Networks:
Parallelizing GenSAT", ``Self-determinations of Man".

\subsubsection{Lemmas, Propositions, and Theorems.}

The numbers accorded to lemmas, propositions, and theorems, etc. should
appear in consecutive order, starting with Lemma 1, and not, for
example, with Lemma 11.

\subsection{Figures}

For \LaTeX\ users, we recommend using the \emph{graphics} or \emph{graphicx}
package and the \verb+\includegraphics+ command.

Please check that the lines in line drawings are not
interrupted and are of a constant width. Grids and details within the
figures must be clearly legible and may not be written one on top of
the other. Line drawings should have a resolution of at least 800 dpi
(preferably 1200 dpi). The lettering in figures should have a height of
2~mm (10-point type). Figures should be numbered and should have a
caption which should always be positioned \emph{under} the figures, in
contrast to the caption belonging to a table, which should always appear
\emph{above} the table; this is simply achieved as matter of sequence in
your source.

Please center the figures or your tabular material by using the \verb+\centering+
declaration. Short captions are centered by default between the margins
and typeset in 9-point type (Fig.~\ref{fig:example} shows an example).
The distance between text and figure is preset to be about 8~mm, the
distance between figure and caption about 6~mm.

To ensure that the reproduction of your illustrations is of a reasonable
quality, we advise against the use of shading. The contrast should be as
pronounced as possible.

If screenshots are necessary, please make sure that you are happy with
the print quality before you send the files.
\begin{figure}
\centering
\includegraphics[height=6.2cm]{eijkel2}
\caption{One kernel at $x_s$ (\emph{dotted kernel}) or two kernels at
$x_i$ and $x_j$ (\textit{left and right}) lead to the same summed estimate
at $x_s$. This shows a figure consisting of different types of
lines. Elements of the figure described in the caption should be set in
italics, in parentheses, as shown in this sample caption.}
\label{fig:example}
\end{figure}

Please define figures (and tables) as floating objects. Please avoid
using optional location parameters like ``\verb+[h]+" for ``here".

\paragraph{Remark 1.}

In the printed volumes, illustrations are generally black and white
(halftones), and only in exceptional cases, and if the author is
prepared to cover the extra cost for color reproduction, are colored
pictures accepted. Colored pictures are welcome in the electronic
version free of charge. If you send colored figures that are to be
printed in black and white, please make sure that they really are
legible in black and white. Some colors as well as the contrast of
converted colors show up very poorly when printed in black and white.

\subsection{Formulas}

Displayed equations or formulas are centered and set on a separate
line (with an extra line or halfline space above and below). Displayed
expressions should be numbered for reference. The numbers should be
consecutive within each section or within the contribution,
with numbers enclosed in parentheses and set on the right margin --
which is the default if you use the \emph{equation} environment, e.g.,
\begin{equation}
  \psi (u) = \int_{o}^{T} \left[\frac{1}{2}
  \left(\Lambda_{o}^{-1} u,u\right) + N^{\ast} (-u)\right] dt \;  .
\end{equation}

Equations should be punctuated in the same way as ordinary
text but with a small space before the end punctuation mark.

\subsection{Footnotes}

The superscript numeral used to refer to a footnote appears in the text
either directly after the word to be discussed or -- in relation to a
phrase or a sentence -- following the punctuation sign (comma,
semicolon, or period). Footnotes should appear at the bottom of
the
normal text area, with a line of about 2~cm set
immediately above them.\footnote{The footnote numeral is set flush left
and the text follows with the usual word spacing.}

\subsection{Program Code}

Program listings or program commands in the text are normally set in
typewriter font, e.g., CMTT10 or Courier.

\medskip

\noindent
{\it Example of a Computer Program}
\begin{verbatim}
program Inflation (Output)
  {Assuming annual inflation rates of 7%, 8%, and 10%,...
   years};
   const
     MaxYears = 10;
   var
     Year: 0..MaxYears;
     Factor1, Factor2, Factor3: Real;
   begin
     Year := 0;
     Factor1 := 1.0; Factor2 := 1.0; Factor3 := 1.0;
     WriteLn('Year  7% 8% 10%'); WriteLn;
     repeat
       Year := Year + 1;
       Factor1 := Factor1 * 1.07;
       Factor2 := Factor2 * 1.08;
       Factor3 := Factor3 * 1.10;
       WriteLn(Year:5,Factor1:7:3,Factor2:7:3,Factor3:7:3)
     until Year = MaxYears
end.
\end{verbatim}
%
\noindent
{\small (Example from Jensen K., Wirth N. (1991) Pascal user manual and
report. Springer, New York)}

\subsection{Citations}

For citations in the text please use
square brackets and consecutive numbers: \cite{jour}, \cite{lncschap},
\cite{proceeding1} -- provided automatically
by \LaTeX 's \verb|\cite| \dots\verb|\bibitem| mechanism.

\subsection{Page Numbering and Running Heads}

There is no need to include page numbers. If your paper title is too
long to serve as a running head, it will be shortened. Your suggestion
as to how to shorten it would be most welcome.

\section{LNCS Online}

The online version of the volume will be available in LNCS Online.
Members of institutes subscribing to the Lecture Notes in Computer
Science series have access to all the pdfs of all the online
publications. Non-subscribers can only read as far as the abstracts. If
they try to go beyond this point, they are automatically asked, whether
they would like to order the pdf, and are given instructions as to how
to do so.

Please note that, if your email address is given in your paper,
it will also be included in the meta data of the online version.

\section{BibTeX Entries}

The correct BibTeX entries for the Lecture Notes in Computer Science
volumes can be found at the following Website shortly after the
publication of the book:
\url{http://www.informatik.uni-trier.de/~ley/db/journals/lncs.html}

\subsubsection*{Acknowledgments.} The heading should be treated as a
subsubsection heading and should not be assigned a number.

\section{The References Section}\label{references}

In order to permit cross referencing within LNCS-Online, and eventually
between different publishers and their online databases, LNCS will,
from now on, be standardizing the format of the references. This new
feature will increase the visibility of publications and facilitate
academic research considerably. Please base your references on the
examples below. References that don't adhere to this style will be
reformatted by Springer. You should therefore check your references
thoroughly when you receive the final pdf of your paper.
The reference section must be complete. You may not omit references.
Instructions as to where to find a fuller version of the references are
not permissible.

We only accept references written using the latin alphabet. If the title
of the book you are referring to is in Russian or Chinese, then please write
(in Russian) or (in Chinese) at the end of the transcript or translation
of the title.

The following section shows a sample reference list with entries for
journal articles \cite{jour}, an LNCS chapter \cite{lncschap}, a book
\cite{book}, proceedings without editors \cite{proceeding1} and
\cite{proceeding2}, as well as a URL \cite{url}.
Please note that proceedings published in LNCS are not cited with their
full titles, but with their acronyms!

\begin{thebibliography}{4}

\bibitem{moore65} Moore, G.E.: Cramming More Components onto Integrated Circuits. (1965)

\bibitem{ms-par} Parallel Programming in the .NET Framework, Microsoft Corp., \url{http://msdn.microsoft.com/en-us/library/dd460693(v=vs.110).aspx}

\bibitem{es-web} EcmaScript, ECMA, \url{http://www.ecmascript.org/}

\bibitem{asm-web} asm.js, \url{http://asmjs.org/}

\bibitem{webgl-web} WebGL, Khronos Group, \url{http://www.khronos.org/webgl/}

\bibitem{rivertrail} Herhut, S., Hudson, R. L., Shpeisman, T., Sreeram, J.: River trail: A path to parallelism in javascript. SIGPLAN Not., 48(10):729--744, (2013).

\bibitem{par-js} Wang, J., Rubin, N., Yalamanchili, S.: River trail: ParallelJS: An Execution Framework for JavaScript on Heterogeneous Systems. SIGPLAN Not. (2014)

\bibitem{js-simd} McCutchan, J., Feng, H., Matsakis, N. D., Anderson, Z., Jensen, P.: A SIMD Programming Model for Dart, JavaScript, and other dynamically typed scripting languages (2014)

\bibitem{w3c-ww} Web Workers, W3C, \url{http://www.w3.org/TR/workers/}

\bibitem{html-whatwg} HTML Standard, WHATWG, \url{https://html.spec.whatwg.org}

\bibitem{jour} Smith, T.F., Waterman, M.S.: Identification of Common Molecular
Subsequences. J. Mol. Biol. 147, 195--197 (1981)

\bibitem{lncschap} May, P., Ehrlich, H.C., Steinke, T.: ZIB Structure Prediction Pipeline:
Composing a Complex Biological Workflow through Web Services. In: Nagel,
W.E., Walter, W.V., Lehner, W. (eds.) Euro-Par 2006. LNCS, vol. 4128,
pp. 1148--1158. Springer, Heidelberg (2006)

\bibitem{book} Foster, I., Kesselman, C.: The Grid: Blueprint for a New Computing
Infrastructure. Morgan Kaufmann, San Francisco (1999)

\bibitem{proceeding1} Czajkowski, K., Fitzgerald, S., Foster, I., Kesselman, C.: Grid
Information Services for Distributed Resource Sharing. In: 10th IEEE
International Symposium on High Performance Distributed Computing, pp.
181--184. IEEE Press, New York (2001)

\bibitem{proceeding2} Foster, I., Kesselman, C., Nick, J., Tuecke, S.: The Physiology of the
Grid: an Open Grid Services Architecture for Distributed Systems
Integration. Technical report, Global Grid Forum (2002)

\bibitem{url} National Center for Biotechnology Information, \url{http://www.ncbi.nlm.nih.gov}

\end{thebibliography}


\section*{Appendix: Springer-Author Discount}

LNCS authors are entitled to a 33.3\% discount off all Springer
publications. Before placing an order, the author should send an email, 
giving full details of his or her Springer publication,
to \url{orders-HD-individuals@springer.com} to obtain a so-called token. This token is a
number, which must be entered when placing an order via the Internet, in
order to obtain the discount.

\section{Checklist of Items to be Sent to Volume Editors}
Here is a checklist of everything the volume editor requires from you:


\begin{itemize}
\settowidth{\leftmargin}{{\Large$\square$}}\advance\leftmargin\labelsep
\itemsep8pt\relax
\renewcommand\labelitemi{{\lower1.5pt\hbox{\Large$\square$}}}

\item The final \LaTeX{} source files
\item A final PDF file
\item A copyright form, signed by one author on behalf of all of the
authors of the paper.
\item A readme giving the name and email address of the
corresponding author.
\end{itemize}
\end{document}
