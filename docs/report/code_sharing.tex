\section{Code sharing}\label{sec:code-sharing}
The library must handle the distribution of the code between the different workers when an operation is executed. This is a challenging problem with many possible alternatives: Cloud Haskell \cite{cloud-haskell} for example, requires a customized version of the Glasgow Haskell Compiler (GHC) to serialize closures and their captured variables as a code pointer and an environment, operating under the assumption that the same code is executed in all nodes. A proposal for using Scheme for distributed programming \cite{distributed-scheme} uses a custom serializable data structure to represent procedures, and macros in order to capture a closure's captured variables and serialize their values. It also assumes that all nodes are executing the same program since serialized functions store pointers to parts of source code files.

In our case, modifying the runtime is not a viable alternative since we want our library to be usable by anyone creating applications for major browser versions.

A naive approach would be trying to send the function to be executed directly to the worker. The following code fragment attempts to do so:
\lstinputlisting[caption=Trying to send a function to a worker, escapechar=]{../../spikes/2.2.1/test-functionSerialization/funcSerialization.js}

When executed in the Google Chrome console the error shown in Figure~\ref{fig:func-serialization} is thrown.
\img{Error when trying to send a \code{Function} object to a Web Worker}{../../spikes/2.2.1/test-functionSerialization/funcSerialization.png}{fig:func-serialization}

That error is expected. As explained in section 9.5.3 of the HTML standard \cite{html-whatwg}, the value being sent must support the structured clone algorithm and a \code{Function} does not meet that requirement as explained in section 2.7.5 of that same standard. It could be assumed that this is because of the challenges of serializing functions.

\subsection{Serializing functions}
Considering the fact that \tfunction{} objects cannot be directly transferred, a different serialization approach is to be considered. Since function closures will not be transferrable a possible alternative is to:
\begin{enumerate}
  \item Serialize the function to a \tstring{}.
  \item Transfer the \tstring{} to the worker.
  \item Create a new function on the worker from that \tstring{}.
\end{enumerate}

\note{An approach to provide contextual data to worker functions is explored in Section~\ref{sec:context}.}

\subsubsection{\tstring{} serialization benchmarks}
One possible way of passing serialized \tfunction{}s to workers would be to encode them as binary and transfer them as an \tabuffer{}. To encode the strings there are two possible approaches:
\begin{itemize}
  \item Use the Encoding API\cite{encoding-api}
  \item Implement non native encoding/decoding functions
\end{itemize}

We created a benchmark to understand the difference between each approach:
\bmcode{http://jsperf.com/pjs-serialization-comparison/4}{../../spikes/2.2.1/test-syncEncodingAPI/encodingAPI.js}

The results for it are displayed in Figure~\ref{fig:string-encode-decode}.
\img{Comparing string encoding/decoding approaches}{../../spikes/2.2.1/test-syncEncodingAPI/encodingAPI}{fig:string-encode-decode}

\subsubsection{Conclusion}
The native implementation was faster than the non-native one and the difference increased between Chrome versions, which seems to indicate that some extra optimizations could be expected in the future.

\subsection{Transferring strings}
Function's code not only needs to be serialized but also transferred. For that reason it was also worth comparing the time it takes to encode/decode a \tstring{} and tranfer the resulting \tabuffer{} to the worker against just passing the \tstring{} to the worker.

Additionally, in most scenarios code would have to be sent along with a \ttarray{} so it would also be interesting to see if the transfer time was affected by this fact.

\subsubsection{\tstring{} serialization and tranfer benchmarks}
We created a benchmark using two functions whose string representation has different lengths to see if the function's length affected the serialization and transfer time. We measured three approaches to transfer the string:
\begin{enumerate}
  \item Encoding the \tstring{} into an \tabuffer{}, sending it as a \code{Tranferable} and decoding it on the Web Worker.
  \item Sending the string directly
  \item Using a \code{Blob} to store the stream and retrieving it in the Worker.
\end{enumerate}

The benchmark with the short function was the following one:
\bmcode{http://jsperf.com/pjs-serialization/5}{../../spikes/2.2.1/test-howToSendCode/shortSerialization.js}

The results for it are displayed in Figure~\ref{fig:short-serialization}.
\img{Comparing string encoding/decoding approaches}{../../spikes/2.2.1/test-howToSendCode/shortSerialization}{fig:short-serialization}

The benchmark with the long function was the following one:
\bmcode{http://jsperf.com/pjs-serialization-long/3}{../../spikes/2.2.1/test-howToSendCode/longSerialization.js}

The results for it are displayed in Figure~\ref{fig:long-serialization}.
\img{Comparing string encoding/decoding approaches}{../../spikes/2.2.1/test-howToSendCode/longSerialization}{fig:long-serialization}

\subsubsection{\tstring{} serialization and shared tranfer benchmark}

We created a benchmark that compares these two scenarios:

\begin{enumerate}
\item Sending the \tstring{} representation of a function to four Web Workers.
\item Encoding a function into a \tsabuffer{} and sending it to four Web Workers.
\end{enumerate}

The code for the benchmark is presented in the following code listing:
\bmcode{http://jsperf.com/shared-function-transfer}{../../spikes/shared/2.2.1/jspBenchmark.longSendCode.js}

The results for it are displayed in Figure~\ref{fig:serialization-vs-shared-transfer}.
\img{Comparing string serialization vs shared buffer transfer approaches}{../../spikes/shared/2.2.1/jspBenchmarklongSendCode.png}{fig:serialization-vs-shared-transfer}

\subsubsection{\tstring{} and \ttarray{} serialization and tranfer benchmarks}
We created a benchmark that tranfers a \ttarray{} and a \tstring{} that are properties of the same object back and forth between the main thread an a worker to understand if transferring additional objects affected the transfer time when comparing it with just tranferring a \tstring{}.

The benchmark's code is the following one:
\bmcode{http://jsperf.com/pjs-encoding/3}{../../spikes/2.2.1/test-howToSendCodeWithData/encoding.js}

The results for it are displayed in Figure~\ref{fig:code-with-data}.
\img{Comparing string encoding/decoding approaches}{../../spikes/2.2.1/test-howToSendCodeWithData/encoding}{fig:code-with-data}

\subsubsection{Conclusion}
As the benchmarks show, the fastest way to tranfer the \tstring{} between the UI thread and workers is by sending it directly.

\pagebreak