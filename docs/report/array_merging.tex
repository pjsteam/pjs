\section{Array merging}

Once all the workers have executed the function provided to them on each element of their respective \ttarray{}s they are to notify the main thread and provide the resulting \ttarray{}.

At this point the main thread must consolidate the results into a new \ttarray{} that represents the result for the entire operation. This means copying all the elements of each partial result into a larger \ttarray{} while maintaining the original order.

\subsection{\ttarray{} merge benchmark}
Unlike the case for \ttarray{} splitting (Section~\ref{subsec:copying-typed-arrays}) there aren't that many native alternatives to create a \ttarray{} from smaller ones.

The benchmark presented only has one alternative that uses a method with a native implementation:
\bmcode{http://jsperf.com/typedarray-merge}{../../spikes/2.4.1/merge.js}

The results for the benchmark are displayed in Figure~\ref{fig:typed-array-merge}.
\img{Comparing \ttarray{} merge approaches}{../../spikes/2.4.1/merge}{fig:typed-array-merge}

\subsection{Conclusion}
The benchmark shows that the best alternative to merge small \ttarray{}s into a larger one is using \code{TypedArray::Set} method.

For illustration purposes, a naive implementation (without optimizations or error checking) would be similar to the following one:
\begin{lstlisting}[caption=Simple \ttarray{} merge function]
function merge(arrays){
  var first = arrays[0];
  var total = arrays.reduce(function(c,a) { return c + a.length; }, 0);
  var result = new first.constructor(total);
  var start = 0;

  arrays.forEach(function(a){
    result.set(array, start);
    start += array.length;
  });

  return result;
}
\end{lstlisting}

\pagebreak