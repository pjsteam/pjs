\section{Array merging}

Once all the workers have executed the function provided to them on each element of their portion of the original array, they are to notify the main thread and provide the resulting \tabuffer{} or \tsabuffer{}.

At this point the UI thread must consolidate the results from the Web Workers into a new \ttarray{} or \tstarray{} that represents the result for the entire operation.\footnote{The case for chains with \code{reduce} as the last step similar and simpler. Thus, it is not worth analyzing.}

For the \tabuffer{} case, this means copying all the elements of each partial result into a larger \ttarray{} while maintaining the original order.

For the \tsabuffer{} case there are two different options:
\begin{enumerate}
  \item If all steps in the chain perform the \code{map} transformation, then no changes are required.
  \item If there is at least one \code{filter} step, the elements need to be placed at the beginning of the resulting \tstarray{} in order to avoid \textif{holes}.
\end{enumerate}

The latter is similar to the \tabuffer{} case, but instead of copying from one array to another, the source and target are the same.

\subsection{\ttarray{} merge benchmark}
Unlike the case for \ttarray{} splitting (Section~\ref{subsec:copying-typed-arrays}) there aren't that many native alternatives to create a \ttarray{} from smaller ones.

The benchmark presented only has one alternative that uses a method with a native implementation:
\bmcode{http://jsperf.com/typedarray-merge}{../../spikes/2.4.1/merge.js}

The results for the benchmark are displayed in Figure~\ref{fig:typed-array-merge}.
\img{Comparing \ttarray{} merge approaches}{../../spikes/2.4.1/merge}{fig:typed-array-merge}

\subsection{\tstarray{} merge benchmark}

The benchmark presented modifies \ttarray{} merge benchmarks to support \tstarray{}s. The implementation removes \dataview{} test case as it is not supported by \tstarray{}:
\bmcode{http://jsperf.com/sharedtypedarray-merge}{../../spikes/2.4.1/sharedMerge.js}

The results for the benchmark are displayed in Figure~\ref{fig:shared-typed-array-merge}.
\img{Comparing \tstarray{} merge approaches}{../../spikes/2.4.1/sharedMerge}{fig:shared-typed-array-merge}

\subsection{Conclusion}
The benchmark shows that the best alternative to merge small arrays into a larger one is using \code{Set} method of eighter \ttarray or \tstarray.

For illustration purposes, a naive implementation (without optimizations or error checking) would be similar to the following one:
\begin{lstlisting}[caption=Simple \ttarray{} merge function]
function merge(arrays){
  var first = arrays[0];
  var total = arrays.reduce(function(c,a) { return c + a.length; }, 0);
  var result = new first.constructor(total);
  var start = 0;

  arrays.forEach(function(a){
    result.set(array, start);
    start += array.length;
  });

  return result;
}
\end{lstlisting}

\pagebreak