\section{Introduction}
In 1965 Gordon Moore proposed the popularly known ``Moore's law" \cite{moore65}, which predicts that the number of transitors in integrated circuits will double every two years.
As a consequence, programs that use a single processor/core will automatically become faster without the need to modify them with that purpose in mind.

In the year 2010 it was predicted that this tendency would be slowly reaching its end, mostly due to reasons related to heat dissipation. This is the reason why newer computers have a greater number of processors instead of processors with more computing power.

To be able to maximize the performance of these computers, it is paramount that systems professionals can create programs that process data in parallel, through different mechanisms, such as threads or processes executing simultaneously in different processors. In the case of applications executed on a single device (e.g. browsers and web pages, applications for mobile devices, desktop apps, server applications) a lot of languages and platforms provide a simple way for programmers to abstract the complexity and coordination required for this type of processing. For example, the .NET platform has the Task Parallel Library (TPL) and Parallel LINQ (PLINQ) \cite{ms-par} libraries that use algorithmic skeletons modelled with high order functions so developers can perform complex operations through a simple API.

On a different note, one of the tendencies that is beginning to take off is the use of EcmaScript \cite{es-web} (also known as Javascript) to create applications, such as video games, that were previously only considered viable in a native environment. This has been possible thanks to components like asm.js \cite{asm-web} and WebGL \cite{webgl-web} together with the constant evolution of browsers and JavaScript runtime engines.
In this context, one of the plans of the commitee that develops the language is to provide an API to simplify the processing of data in parallel for version 7 (ES7) of the language, with the goal of keeping up with native applications. Initiatives such as ParallelJS \cite{par-js} and River trail \cite{rivertrail} or the possibility of taking advantage of SIMD instructions \cite{js-simd} (single instruction multiple data) are some of the options to implement this.

The goal of this paper is to provide an alternative for parallel code execution in EcmaScript, through a library that exposes high order functions to model algorithmic skeletons such as map, filter and reduce. This library will take advantage of Web Workers \cite{w3c-ww} (the mechanism proposed by browsers for parallelism at the time of writing) to allow the simultaneous execution of code in multiple processors.

\pagebreak