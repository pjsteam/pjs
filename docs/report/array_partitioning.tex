\section{Array partitioning}

As explained in Section~\ref{sec:serialization-and-transference} of this document, transferring \ttarray{} objects to and from Web Workers requires the use of \code{Transferable}s to achieve acceptable performance. Since our library will receive \ttarray{} objects to be partitioned and handed over to multiple workers, it is important that it spends as little time as possible partitioning the arrays.\footnote{This does not apply to \tstarray{}s as those can be transferred without being partitioned.}

If the library is configured to use \(N\) workers then the original data in the \ttarray{} will copied into \(N\) smaller \ttarray{}s. For ideal performance it would be nice to be able to pass a \textit{pointer} into the array and an amount of elements to process, but because \ttarray{}s do not support shared memory, and element ownership is tranferred to the Web Workers, the elements must be copied to the sub-arrays.

\subsection{Copying \ttarray{}s}
\label{subsec:copying-typed-arrays}
Sections 22.2 and 24.1 of the ECMAScript 6 draft \cite{es6} specify the API for \ttarray{} and \tabuffer{} objects respectively. Some of the exposed operations provide a way to copy the contents of an \tabuffer{} into another \tabuffer{}, so they are to be considered as viable alternatives for the copy operation. Other alternatives to consider are:
\begin{itemize}
  \item Implementing a non-native function that copies the array elements one by one.
  \item Using a \code{Blob} to store the contents of a \ttarray{} and reading the \code{Blob}'s contents into a new \ttarray{}.
\end{itemize}

\subsubsection{\ttarray{} copy benchmark}
A benchmark was created based on the aforementioned alternatives:
\bmcode{http://jsperf.com/arraybuffer-copy/4}{../../spikes/2.3.1/copyComparison.js}

The results for it are displayed in Figure~\ref{fig:typed-array-copy}.
\img{Comparing \ttarray{} copy approaches}{../../spikes/2.3.1/copyComparison}{fig:typed-array-copy}

Based on those results we can discard the \code{Blob} approach for partitioning \ttarray{}s.

\note{Alternatives \textit{Buffer subarray} and \textit{Constructor} do the same thing in the case of copying an entire \ttarray{}. However, \textit{Constructor} is not a viable alternative for copying part of an array}

\subsubsection{\ttarray{} partition benchmark}
Based on the result from the previous benchmark we created a new benchmark to understand which alternative is better to partition an array:
\bmcode{http://jsperf.com/arraybuffer-split/3}{../../spikes/2.3.1/splitComparison.js}

The results for it are displayed in Figure~\ref{fig:typed-array-split}.
\img{Comparing \ttarray{} partition approaches}{../../spikes/2.3.1/splitComparison}{fig:typed-array-split}

\subsubsection{Conclusion}
The benchmark shows that there are three very similar alternatives to partition a \ttarray{}. We decided to use \textit{Buffer subarray}. The implementation of the slice function could look something like this:
\begin{lstlisting}[caption=Possible approach to copy a slice of a \ttarray{}]
function typedArraySlice(array, from, to) {
  return new array.constructor(array.subarray(from, to));
}
\end{lstlisting}

\pagebreak