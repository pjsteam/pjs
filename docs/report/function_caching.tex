\section{Function caching}

Before comparing the sequential implementation of \code{map} against the parallel one we believed that the cost of dynamically instantiating a \tfunction{} from a \tstring{} in the Web Workers could be avoided by introducing a cache for cases where the worker has already seen the code for a particular function.

We decided to introduce a simple key/value map acting as a cache where the keys are the \tstring{}s with the function's code and the values are the \tfunction{} objects.

To verify its performance we created a benchmark that runs both with and without the function cache:
\bmcode{http://jsperf.com/p-j-s-with-vs-without-function-cache}{../../spikes/3.1.3/functionCache/functionCache.js}

The results for the benchmark are displayed in Figure~\ref{fig:function-cache-result}.
\img{Comparing p-j-s \code{map} with and without a function cache}{../../spikes/3.1.3/functionCache/functionCache}{fig:function-cache-result}

\subsection{Conclusion}
As the benchmark shows versions 40 and 41 of chrome have almost no difference between the implementations with and without the cache. To determine whether introducing the cache was worth it or not we also used Chrome Canary version 42. The latter shows a big increase in performance (near 20\%), which is why we decided to leave the \tfunction{} cache as part of our implementation.

\pagebreak