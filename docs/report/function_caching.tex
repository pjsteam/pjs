\section{Function caching}\label{sec:func-caching}

Before comparing the sequential implementation of \code{map} against the parallel one, we believed that the cost of dynamically instantiating a \tfunction{} from a \tstring{} in the Web Workers could be avoided by introducing a cache for cases where the Web Worker has already seen the code for a particular function.

We decided to introduce a simple key/value map acting as a cache where the keys are \tstring{}s with the function's code and the values are the \tfunction{} objects.

To verify its performance we created a benchmark that runs both with and without the function cache:
\bmcode{http://jsperf.com/p-j-s-with-vs-without-function-cache/3}{../../spikes/3.1.3/functionCache/functionCache.js}

The results for the benchmark are displayed in Figure~\ref{fig:function-cache-result}.
\img{Comparing p-j-s \code{map} with and without a function cache}{../../spikes/3.1.3/functionCache/functionCache}{fig:function-cache-result}

\subsection{Conclusion}
As the benchmark shows, there is almost no difference in versions 40 and 41 of Google Chrome between the implementations with and without the cache. To determine whether introducing the cache was worth it or not we considered the result from Firefox Nightly, which shows a big increase in performance (near 25\%). That is why we decided to make the \tfunction{} cache a part of our implementation.

\pagebreak