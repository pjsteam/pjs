\section{Modules system}

\note{At the time of writing, neither Google Chrome nor Mozilla Firefox support the ES6 module specification so that is not a viable choice.}

The two most known module formats/API in JavaScript are known as:
\begin{description}
\item[CommonJS] Recognized for being used in node.js~\cite{nodejs}.
\item[AMD~\cite{amd}] The most known implementation being require.js~\cite{requirejs}.
\end{description}

CommonJS has many similar aspects with the ES6 module specification, although it does not provide a way to load modules asynchronously. AMD is very different from the ES6 module specification but does support asynchronous module loading which is useful for browser applications.

Instead of making the choice for users, our library will support both systems in a way that makes it easier for users to consume either of them.

\subsection{CommonJS support}
All that needs to be done to use the library is shown in the following code snippet:
\begin{lstlisting}[language=HTML, caption=CommonJS usage]
<script src="https://rawgit.com/pjsteam/pjs/v1.0.0-beta/dist/p-j-s.min.js"></script>

<script>
  var pjs = require('p-j-s');
  pjs.init();

  /* use below */
</script>
\end{lstlisting}

\subsection{AMD support}
This example shows how to use the library with require.js. It assumes a folder structure like the one shown below:
\begin{lstlisting}[language=bash, caption=AMD folder structure]
.
|__ index.html
|__ scripts
    |__ main.js
    |__ require.js
\end{lstlisting}

The file \textbf{main.js} must have the following contents:
\begin{lstlisting}[caption=main.js AMD support]
require(["https://rawgit.com/pjsteam/pjs/v1.0.0-beta/dist/p-j-s-standalone.min.js"], function(pjs) {
  pjs.init();

  /* use below */
});
\end{lstlisting}

\subsection{Conclusion}
As the previous examples show, using our library with either module system is really simple and is done in a way that should feel familiar for users of those module formats.

\pagebreak