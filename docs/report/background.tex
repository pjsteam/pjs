\section{Background}\label{sec:background}
All modern versions of major browsers (Internet Explorer, Mozilla Firefox, Google Chrome, Safari and Opera) allow the execution of JavaScript code in a single threaded event loop runtime \cite{event-loop}. User actions add elements to the event loop queue and these are processed sequentially. All process input/output is asynchronous to prevent the UI thread from blocking, thus keeping it responsive.
While such a runtime is a good fit for most web applications, JavaScript and the browser are now also being used for some CPU intensive tasks, such as games, which commonly require physics calculations, image manipulation (in 2D) and graphics generation (in 3D). Despite the large improvement in JavaScript's performance, obtained through optimizing compilers such as Google Chrome's V8 \cite{v8} and Mozilla Firefox's SpiderMonkey \cite{spider-monkey}, it is important to be able to take advantage of multiple cores in modern devices to achieve even greater performance.\\
In 2010, Web Workers were made part of the web standard. Web Workers allow the creation of ``thread like'' constructs in a browser environment, but they don't allow shared memory and instead communicate via message passing. The message passing overhead is very big for common objects and Web Workers ``have a high start-up performance cost'' \cite{w3c-ww-startup}, so they were commonly considered for long running tasks and operating on generally the same set of data during a single execution, making them unfit for a thread pool  model \cite{thread-pool}.\\
However, a spec is being developed for version 7 of the EcmaScript standard that changes this, by introducing shared memory through \tsabuffer{}s \cite{sab}. In essence, the proposal allows the same memory to be shared across multiple Web Workers, and also aims to provide the necessary atomic/lock constructs to deal with shared memory. In the particular case of embarrassingly parallel computations (such as map, filter and reduce operations on an array), it is simple to take advantage of the speed benefits of shared memory without incurring in any overhead due to synchronization.

We developed our library to work with instances of the already standarized \tabuffer{} type using the Google Chrome broswer. Additionally, we support the experimental \tsabuffer{}, which at the time of writing is only available in Firefox Nightly\footnote{Current version is 41.0a1}.

\pagebreak